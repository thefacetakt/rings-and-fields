\documentclass[../main.tex]{subfiles}

\begin{document}
\section{Вопросы на 5}

\begin{enumerate}

\item Для заданных элементов $z_1$, $z_2$ найдите порождающий элемент идеала $(z_1, z_2)$.

\item Найдите факторкольцо $D/(x_1)$, где $N(x_1) < 20$.

\item Простой элемент области целостности является неразложимым.

Пусть $p$ --- простой и $p = xy \Rightarrow x | p \land y | p$ . Из определения простоты $p | x \lor p | y$. Но тогда или $x | p \land p | x$, или $y | p \land p | y$. Тогда $p \sim y \lor p \sim x \Rightarrow y \in K^* \lor x \in K^*$, то есть $p$ --- неразложимый.

\item Для любого числа $u \in \mathbb{C}$ определим множество $\mathbb{Z}[u] = \cup_{n = 0}^{\infty} \{a_0 + a_1u + \ldots + a_nu^n | a_0, a_1, \ldots, a_n \in \mathbb{Z}\}$.

а) Докажите, что $\mathbb{Z}[u]$ является областью целостности.

То, что $\mathbb{Z}[u]$ кольцо проверяется непосредственно. Поскольку $\mathbb{Z}[u] \subset \mathbb{C}$ и $\mathbb{C}$ --- область целостности (\textit{потому что $\mathbb{C}$ --- поле}), то и $\mathbb{Z}[u]$ область целостности.

б) При каких $u \in \mathbb{C}$ данное $\mathbb{Z}[u]$ <<конечномерно над $\mathbb{Z}$>>, то есть найдётся такое $N$, что $\mathbb{Z}[u] = \cup_{n = 0}^{\infty} \{a_0 + a_1u + \ldots + a_nu^N | a_0, a_1, \ldots, a_N \in \mathbb{Z}\}$?

Покажем, что $\mathbb{Z}[u]$ <<конечномерно над $\mathbb{Z}$>>, $ \Leftrightarrow \exists f \in \mathbb{Z}[x]: f(u) = 0, f \ne 0$.

$\Rightarrow$

    Поскольку $u^{N + 1} \in \mathbb{Z}[u] \Rightarrow \exists a_0, \ldots, a_N \in \mathbb{Z}: u^{N + 1} = \sum\limits_{0}^{N} a_ku^k \Rightarrow u$ --- корень $f(x) = x^{N + 1} - \sum\limits_{0}^{N} a_kx^k$

$\Leftarrow$

    Пусть $u$ --- корень многочлена $f(x) = u^{N} + \sum\limits_{0}^N a_kx^k$ (всегда можем поделить на старший коэффициент). Тогда $u^N$ выражается через меньшие степени. ($u^N = \sum\limits_{0}{N - 1} -a_ku^k$)

    Индукцией по $k \geqslant N$ легко показать, что $u^k$ выражается через $1, u, \ldots u^{N - 1}$.

    ($u^{k + 1} = u \cdot u^{k} \stackrel{\textup{предположение индукции}}{=} u \cdot (\sum\limits_0^{N - 1} b_ku^k) = (\sum\limits_1^{N - 1} b_{k - 1} u^k) + b_{N - 1}u^N \stackrel{\textup{база индукции}}{=} (\sum\limits_1^{N - 1} b_{k - 1} u^k) + b_{N - 1}\sum\limits_{0}^{N - 1}-a_ku^k$

\item Пусть $I \subset K$ --- идеал. Всегда ли радикал $\sqrt{I} = \{a \in K | \exists m \in \mathbb{N}: a^m \in I\}$ идеала $I$ является идеалом?

$a \in I, b \in \sqrt{I} \Rightarrow \exists n, m: a^n \in I, b^m \in I \Rightarrow \forall n' \geqslant n, m' \geqslant m \forall x \in K xa^{n'} \in I, xb^{n'} \in I$ (т.к $I$ --- идеал). $(a + b)^{m + n} = \sum\limits_{k = 0}^{m + n} C_{m + n}^{k} a^kb^{m + n - k}$. В каждом слагаемом предыдущей суммы либо степень $a$ не меньше $n$, либо степень $b$ не меньше $m$, то есть каждое слагаемое лежит в $I$ то есть вся сумма лежит в $I$, то есть $(a + b)^{n + m} \in I \Rightarrow (a + b) \in \sqrt{I}$.

Пусть $x \in K \Rightarrow (xa)^n = x^na^n \in I \Rightarrow xa \in \sqrt{I}$.

То есть $\sqrt{I}$ действительно идеал.


\item Докажите, что в кольце главных идеалов любая возрастающая цепочка идеалов

$$ (a_1) \subset (a_2) \subset \ldots \subset (a_n) \subset \ldots $$

стабилизируется, то есть найдется такое $k$, то $(a_k) = (a_{k + 1}) = \ldots$


Поскольку $(a_i) \subset (a_{i + 1}) \Rightarrow a_{i + 1} | a_i$.


Возьмем $I = \cup_{k = 1}^{\infty} (a_k)$. покажем, что $I$ -- идеал. Пусть $a \in I, b \in I \Rightarrow \exists k_1, k_2: a \in (a_{k_1}), b \in (a_{k_2})$. Тогда положим $k = max(k_1, k_2)$. $a, b \in (a_k) \Rightarrow (a + b) \in (a_k) ((a_k) \textup{--- идеал}) \Rightarrow (a + b) \in I$. Анологично $\forall x \in K xa \in (a_k) \Rightarrow xa \in I$.

Поскольку $K$ --- КГИ, то существует $x: I = (x)$. $x \in I \Rightarrow \exists k: x \in (a_k)$. Но $a_k \in (x)$. Тогда $x | a_k \land a_k | x \Rightarrow x \sim a_k$. Но в силу вложенности это верно и для всех $j > k$, то есть $\forall j \geqslant k a_j \sim a_k \Rightarrow (a_j) = (a_k)$. То есть цепочка действительно стабилизируется.

\item $K$ --- евклидово кольцо. Верно ли, что если для $a, b \ne 0$ выполнено равенство $N(ab) = N(a)$, то $b$ обратим?

%$N(a) = N(abb^{-1}) \geqslant N(ab) \geqslant N(a) \Rightarrow

Поделим $a$ с остатком на $ab$:

$$a = abq + r: r=0 \lor N(r) < N(ab)$$.
$$r = a(1 - bq)$$.

Если $r=0$, то $bq = 1$ и $b$ обратим. Иначе $N(ab) > N(r) = N(a(1 - bq)) \geqslant N(a) = N(ab)$. Противоречие.


\item Если $z \in D$, $z | x$, и $N(z) = N(x)$, то $z \sim x$.

Пусть $x = yz$. Тогда $N(yz) = N(z) \Rightarrow y$ обратим (по предыдущей задаче) и, значит, $x \sim z$.

\item а) Если $z$ --- неразложимый элемент $D$, то существует такое простое целое число $p$, что $N(z) = p$ или $N(z) = p^2$

$N(z) = z\overline{z}$. Разложим $N(z)$ в произведение простых как натуральное число:

$z\overline{z} = N(z) = p_1^{\alpha_1} \cdot \ldots \cdot p_n^{\alpha_n}$.

Так как $z$ неразложим, а $D$ --- евклидово, то $z$ --- прост, значит $\exists k: z | p_k$.

$p_k = zu \Rightarrow p_k = \overline{p_k} = \overline{z}\overline{u} \Rightarrow \overline{z} | p_k \Rightarrow N(z) | p_k^2 \Rightarrow N(z) = 1, p_k$ или $p_k^2$. Но так как если $N(z) = 1$, то $z$ --- обратим (а, следовательно, неразложим), то $(z) = p_k \lor N(z) = p_k^2$.


б) Если $z$ — неразложимый элемент $D$ и $N(z) = p^2$, то $z \sim p$.

Пусть $\overline{z} = ab \Rightarrow z = \overline{a} \overline{b} \Rightarrow \overline{z}$ --- неразложим.

$z \overline{z} = N(z) = p \cdot p$. В силу единственности разложения на неразложимые, $z \sim p$.

в) Если $N(z) = p$, то $z$ --- неразложимый элемент $D$.

в $D a|b \Rightarrow N(a) | N(b)$.

Пусть $a | z \Rightarrow N(a) | N(z)$. В силу простоты $N(z)$ либо $N(a) = 1$ и, следовательно, $a$ --- обратимый, либо $N(a) = N(z)$ и тогда $a \sim z$. То есть $z$ неразложим.


г) Пусть $p$ — простое целое число. Тогда есть два варианта: либо $p$ неразложимо в $D$, либо $p$ = $z\overline{z}$, где $z$ -- неразложимо в $D$. Таким образом описываются все неразложимые элементы $D$.


Пусть $p$ разложимо в $D$. Тогда найдется такой неразложимый $z: z|p$. Поскольку $z$ не ассоциирован с $p$, $N(z) \ne N(p) \Rightarrow N(z) = p$. Тогда $z$ -- неразложимый и $z\overline{z} = N(z) = p$.

Любой неразложимый элемент $D$ --- либо простое целое число, либо его норма --- простое целое число.

\item (Простые гауссовы числа) Пусть $p$ --- простое целое число.

а) Если $p$ = $4k + 3$, то $p$ --- неразложим в $\mathbb{Z}[i]$.

Если $p$ разложим, тогда $p = z\overline{z} = Re^2z + Im^2z$. Но число, дающее остаток 3 при делении на 4 не быть представлено в виде суммы двух квадратов (квадраты дают остаток 1 при делении на 4).

б) Если $p = 4k + 1$, то $p$ — разложим в $\mathbb{Z}[i]$.

Если $p = 4k + 1$, то $-1$ --- вычет по модулю $p$, т. е $\exists x \in \mathbb{Z}: p| x^2 + 1 \Rightarrow p | (x + i)(x - i)$. Если $p$ --- неразложим, тогда $p$ --- прост и или $p| (x + i)$, или $p | (x - i)$. В любом случае, т.к $x$ --- целое в силу задачи 18 из задач на 3-4 $p | 1$, что плохо. Значит, $p$ разложим.

в) Если $p = 4k + 1$, то $p = z\overline{z}$, где $z$ — неразложим в $\mathbb{Z}[i]$.

Следует из предыдущего пункта и пункта г) предыдущей задачи.

г) Неразложимые элементы $\mathbb{Z}[i]$, не описанные в предыдущих пунктах --- $\pm 1 \pm i$.

Неразложимые элементы, не описанные в предыдущих задачах могут иметь норму или 2, или 4. Норму 4 имеет только $2$ и ассоциированные с ней, но $2 = (1 + i)(1 - i)$.

С другой стороны, $N(\pm 1 \pm i) = 2$, то есть силу пункта в) предыдущей задачи $\pm 1 \pm i$ неразложимы.


\item (Простые числа Эйзенштейна) Пусть $p$ --- простое целое число.

а) Если $p = 3k + 2$, то $p$ --- неразложим в $\mathbb{Z}[\omega]$.

б) Если $p = 3k + 1$, то $p$ --- разложим в $\mathbb{Z}[\omega]$.

в) Если $p = 3k + 1$, то $p = z\overline{z}$, где $z$ --- неразложим в $\mathbb{Z}[\omega]$.

% 12. а) Для произвольных элементов x1, . . . , xn кольца K множество

% 2

% .

% 2

% , то z ассоциировано с p.

% (x1, . . . , xn) = {a1x1 + . . . + anxn | a1, . . . , an ∈ K}

% является идеалом. Он называется идеалом, порождённым элементами x1, . . . , xn.

% б) Докажите, что (x1, . . . , xn) — минимальный по включению идеал, содержащий элементы x1, . . . , xn.

% 13. Всегда ли множество делителей нуля (с добавлением 0) является идеалом?

% 14. Пусть N(K) — множество нильпотентных элементов кольца K, то еcть элементов, некоторая сте-

% пень которых равна 0 N(K) = {a ∈ K | ∃n ∈ N : a

% Всегда ли N(K) является идеалом?

% 15. а) Докажите, что a | b тогда и только тогда, когда (b) ⊂ (a). б) Докажите, что a ∼ b тогда и

% только тогда, когда (a) = (b).

% 16. Пусть I, J ⊂ K — идеалы. Докажите, что сумма I + J = {x + y | x ∈ I, y ∈ J} и пересечение I ∩ J

% идеалов являются идеалами.

% 17. Пусть K = Z[i], x, y ∈ Z. Докажите, что x и y взаимно-просты в Z тогда и только тогда, когда x и y

% взаимно просты в Z[i].

% 18. Пусть K = Z[i], (x, y) = 1. Какие значения может принимать (x + yi, x − yi)?

% 19. Найдите степень поля разложения для многочленов (например, x

% 20. Найдите примитивный элемент расширения (например Q(

% 21. Многочлен Φp(x) = x

% 22. Верно ли, что любое конечное расширение поля является алгебраическим?

% 23. Верно ли, что у любого поля существует не алгебраическое расширение?

% 24. Если α является корнем неприводимого многочлена f(x), то f(x) порождает идеал

% mα,F = {g(x) ∈ F[x] | g(α) = 0}.

% 25. Пусть f(x), g(x) — неприводимые многочлены со старшим коэффициентом 1, у которых α является

% корнем. Тогда f(x) = g(x).

% 26. Верно ли, что если у K нет нетривиальных алгебраических расширений, то полем разложения любого

% многочлена f ∈ K[x] является K? (Если верно, то докажите это, если неверно, приведите контрпример)

% n = 0}.

% √

% 2); Q(

% 2 − 2, x

% 3 − 2, x

% √

% √

% 3); Q(

% 2,

% √

% p−1 + x

% p−2 + . . . + x + 1 (p — простое число) неприводим над Q.

% 27. Верно ли, что если полем разложения любого многочлена f ∈ K[x] является K, то у K нет нетри-

% виальных алгебраических расширений?

% 28. Верно ли, что если любой многочлен степени > 1 из K[x] имеет корень в поле K, то у K нет

% нетривиальных алгебраических расширений?

% 29. Верно ли, что если у K нет нетривиальных алгебраических расширений, то любой многочлен степени

% > 1 из K[x] имеет корень в поле K?

% 30. Верно ли, что над алгебраически замкнутым полем K нет нетривиальных расширений?

% 31. Пусть F ⊂ K — расширение полей, H ⊂ AutF (K) — подгруппа. Тогда KH = {x ∈ K | ∀h ∈ H h(x) = x}

% является полем, причём K ⊃ KH ⊃ F.

% 32. Докажите, что если K ⊃ F — расширение Галуа степени n, то AutF K вкладывается в группу Sn.

% 33. Для конечного поля: построение, нахождение порождающего элемента поля.
\end{document}
