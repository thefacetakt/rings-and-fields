\section{Определения}

\begin{enumerate}

\item Определение коммутативного кольца.

\item Определение обратимого элемента в кольце.

\item Определение делителя нуля в кольце.

\item Определение гомоморфизма колец.

\item Определение области целостности.

\item Определение элемента кольца, ассоциированным с данным элементом.

\item Определение неразложимого элемента области целостности.

\item Определение простого элемента области целостности.

\item Определение евклидова кольца.

\item Определение факториального кольца.

\item Определение чисел Эйзенштейна.

\item Определение наибольшего общего делителя двух элементов области целостности.

\item Определение подкольца в кольце.

\item Определение идеала в кольце.

\item Какие идеалы в кольце называются тривиальными?

\item Определение идеала, порождённого элементами $x_1, \ldots, x_n$.

\item Определение конечно порождённого идеала.

\item Определение главного идеала.

\item Определение кольца главных идеалов.

\item Определение простого идеала.

\item Определение максимального идеала.

\item Определение примитивного многочлена.

\item Определение расширения поля.

\item Определение алгебраического элемента расширения поля.

\item Определение трансцендентного элемента расширения поля.

\item Определение алгебраического расширения поля.

\item Пример алгебраического расширения поля.

\item Пример не алгебраического расширения поля.

\item Определение минимального многочлена элемента расширения поля. (3 условия)

\item Пуcть $K \supset F$  --- расширение, $\gamma \in K$ --- элемент. Что такое $F(\gamma)$?

\item Определение поля разложения многочлена

\item Определение алгебраически замкнутого поля (все 4 условия).

\item Определение алгебраического замыкания поля.

\item Что значит построить элемент $z \in C$ при помощи циркуля и линейки.

\item Определение $\xi_n$ --- примитивного корня $n$-ой степени из $1$.

\item Определение сопряжённого элемента к данному элементу из $K \supset F$.

\item Сформулируйте критерий неприводимости Эйзенштейна.

\item Определения группы автоморфизмов $Aut_{F}(K)$ расширения $K \supset F$.

\item Определения поля $K^H$ для подгруппы автоморфизмов $H \subset Aut_F(K)$ расширения $K \supset F$.

\item Определение расширения Галуа/нормального сепарабельного расширения (4 равносильных условия).

\item Определение группы Галуа расширения.

\item Что означает разрешимость уравнения $f(x) = 0$ в радикалах? (Здесь $f(x)$ --- многочлен над полем $F$).

\end{enumerate}
