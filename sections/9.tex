\documentclass[../main.tex]{subfiles}

\begin{document}

\setcounter{section}{8}
\section{Вопросы на 9}

    \hypertarget{9.1}{\subsection{Эквивалентность определений нормального
    сепарабельного расширения (расширения Галуа)}}.

    Пусть $K \supset F$ --- конечное сепарабельное расширение.
    Тогда следуещие условия эквивалентны:

    \begin{enumerate}
        \item Для любого элемента $\alpha \in K$, любой сопряженный к $\alpha$
        над $F$ тоже лежит в $K$

        \item $K$ является полем разложения какого-либо многолена над $F$

        \item $|Aut_FK| = [K : F]$

        \item $K^{Aut_FK} = F$
    \end{enumerate}

    Такое расширение называется \textit{нормальным}
    или \textit{расширением Галуа}.

    $1 \Rightarrow 2$

    Так как расширение конечное, $K = F(\alpha_1, \ldots, \alpha_n)$.

    Положим $f := m_{\alpha_1, F} \cdot \ldots \cdot m_{\alpha_n, F}$.
    Тогда, поскольку все сопряженные к $\alpha_1, \ldots, \alpha_n$
    лежат в $K$, $K$ содержит все корни $f$. С другой стороны, если
    $K \supset L$ содержит все корни $F$, то $\alpha_1, \ldots, \alpha_n
    \in L \Rightarrow F(\alpha_1, \ldots, \alpha_n) \subset L \Rightarrow
    K \subset L \subset K \Rightarrow L = K$,
    то есть $K$ --- поле разложения $f$ над $F$.

    $2 \Rightarrow 3$

    \textbf{Утверждение 1.} Любой гомоморфизм $\varphi K \to \overline{F}$,
    сохраняющий $F$ переводит элементы $K$ в сопряженные к ним над $F$.

    Доказательство утверждения 1:

    Пусть $\alpha \in K, m_{\alpha, F} = \sum\limits_{k=0}^n a_kx^k.
    m_{\alpha, F}(\alpha) = 0 \Rightarrow \varphi(m_{\alpha, F}(\alpha))
    = \varphi(0) = 0$.

    С другой стороны $0 = \varphi(m_{\alpha, F}(\alpha))
    = \varphi(\sum\limits_{k=0}^n a_k \alpha^k)
    = \sum\limits_{k=0}^n a_k \varphi(\alpha)^k
    = m_{\alpha, F}(\varphi(\alpha))$,
    что и означает, что $\varphi(\alpha)$ сопряжен к $\alpha$ над $F$.

    \hypertarget{9.1.statement.2}{\textbf{Утверждение 2.}}
    Пусть $\varphi: K \to \overline{F}$
    --- гомоморфизм, сохраняющий $F$.
    Тогда $\varphi$ является автоморфизмом $K$.

    Действительно, пусть $K$ --- поле разложения $f$ над $F$, и
    $\alpha_1, \ldots, \alpha_n$ --- корни $f$.

    Тогда $K = F(\alpha_1, \ldots, \alpha_n)$.

    Поскольку для любого $i: m_{\alpha_i, F} | f \Rightarrow$
    то все сопряженные к $\alpha_i$ над $F$ находятся среди корней $f$.

    По утверждению 1 множество $\{\alpha_1, \ldots, \alpha_n\}$ переходит в свое
    подмножество, а учитывая, что любой нетривиальный гомоморфизм полей
    инъективен, то на самом деле оно переходит само в себя (в силу конечности).
    Тогда $\varphi$ задает на множестве индексов корней $f$ некую перестановку
    $\sigma$.

    Пусть $\beta \in K, \beta = \sum\limits_{k=0}^n a_k \alpha_k, a_k \in F$.
    Тогда $\varphi(\beta) = \varphi(\sum\limits_{k=0}^n a_k \alpha_k) =
    = \sum\limits_{k=0}^n a_k \alpha_{\sigma(k)} \in K$.

    То есть $\varphi(K) \subset K$.

    С другой стороны $\varphi(\sum\limits_{k = 0}^n a_k \alpha_{\sigma^{-1}(k)})
    = \sum\limits_{k = 0}^n a_k \alpha_k = \beta$. То есть $\varphi(K) = K$.

    Итого, $\varphi:K \to K$ --- сюръективный гомоморфизм полей, а значит
    --- автоморфизм $K$.

    Поскольку $K \supset F$ --- конечное сепарабельное, то по теореме о
    примитивном элементе найдется такое $\gamma$, что $K = F(\gamma)$.

    Пусть $\gamma = \gamma_1, \gamma_2, \ldots, \gamma_m$ --- корни
    $m_{\gamma, F}$.

    Вспомним утверждение \hyperlink{6.13}{6.13} (точнее, его доказательство).

    $K = F(\gamma_1) \stackrel{\varphi}{\cong} F(\gamma_i)$, причем $\varphi$
    сохраняет $F$ и $\varphi(\gamma_1) = \gamma_2$.

    $\varphi: K \to \overline{F}$ --- гомоморфизм, сохраняющий $F$,
    следовательно, по утверждению 2 он является автоморфизмом $K$, сохраняющем
    $F$. То есть для любого $i$ существует $\varphi \in Aut_FK:
    \varphi(\gamma_1) = \gamma_i \Rightarrow |Aut_FK| \geqslant m$.

    С другой стороны, тем, куда переходит $\gamma = \gamma_1$ автоморфизм,
    сохраняющий $F$ полностью определяется (поскольку любой элемент $K$
    разлагается по степеням $\gamma$ с коэффициентами из $F$). Значит,
    $Aut_FK \leqslant m$. Значит, $Aut_FK = m = \deg m_{\gamma, F} = [K : F]$,
    что и требовалось доказать.

    $3 \Rightarrow 4$

    Пусть $K^{Aut_FK} = L. K \supset L \supset F$ (\hyperlink{5.31}{5.31})

    Пусть, по-прежнему, $K = F(\gamma)$. Мы уже выяснили, что при автоморфизме
    $K$, сохраняющем $F$ $\gamma$ переходит в корень $m_{\gamma, F}$, причем
    тем, куда переходит $\gamma$ полностью определяется автоморфизм.

    Заметим, что $K = L(\gamma)$, и что все вышесказанное справедливо и для
    расширения $K \supset L$, то есть $|Aut_LK| \leqslant \deg m_{\gamma, L}
    = [K : L]$

    Все автоморфизмы, сохраняющие $F$ сохраняют и $L$ (по определению $L$),
    значит, $|Aut_FK| \leqslant |Aut_LK| \leqslant [K : L] \leqslant [K : F]$.

    Но $|Aut_FK| = [K : F] \Rightarrow [K : L] = [K : F] \Rightarrow L = F$.

    $4 \Rightarrow 1$

    \hypertarget{9.1.useful.statement}{Рассмотрим вспомогательное утверждение:}

    Пусть $K^H = F$. Тогда для любого $\beta \in K: |H| \geqslant
    m_{\alpha, F}$ (это нам конкретно сейчас не понадобится)
    и любой сопряженный к $\beta$ над $F$ лежит в $K$
    (а вот это будем использовать).

    Докажем его. Рассмотрим $f_\beta
    = {\displaystyle \prod_{h \in H}(x - h(\beta))}$.

    Рассмотрим действие элементами $H$ на элементах
    $K[x]:$

    $H \ni h \mapsto \alpha_h(\sum a_k x^k) = \sum h(a_k) x^k$.

    Проверим, что это действие
    (напомню: действие, это гомомофизм из $H$ в группу биекций $K[x]$).

    1) Инъективность:

    Пусть $\alpha_h(g_1) = \alpha_h(g_2)$, тогда образы всех
    коэффициенто $g_1$ совпадают с образами всех коэффициентов $g_2$.
    Но $h$ --- автоморфизм, так что все \textit{коэффициенты} $g_1$ совпадают
    с коэффициентами $g_2$

    2) Сюръективность:

    $\alpha_h(\sum h^{-1}(\alpha_k) x^k) = \sum \alpha_kx^k$

    3) Гомоморфность:

    $\alpha_{h_1} \circ \alpha_{h_2}(\sum a_k x^k)
    = \alpha_{h_1}(\sum h_2(a_k) x^k) = \sum h_1 h_2 (a_k) x^k
    = \alpha_{h_1 h_2}$

    Заметим также, что $\alpha_h((\sum a_k x^k)(\sum b_kx^k))
    = \alpha_h(\sum (\sum\limits_{i + j = k} a_ib_j)x^k)
    = \sum h(\sum\limits_{i + j = k} a_ib_j)x^k
    = \sum (\sum\limits_{i + j = k} h(a_i)h(b_j)) x^k
    = (\sum h(a_k)x^k)(\sum h(b_k)x^k)
    = \alpha_h(\sum a_kx^k)\alpha_h(\sum b_kx^k)$.

    Иными словами, $\alpha_h(fg) = \alpha_h(f)\alpha_h(g)$.

    Возьмем произвольный $g \in H$. Учитывая вышесказанное

    $\alpha_g(f_\beta) = \prod\limits_{h \in H} \alpha_g(x - h(\beta))
    =  \prod\limits_{h \in H} (x - gh(\beta))$. Но умножение на элемент
    группы есть автоморфизм группы, то есть $gH = H$, то есть
    $\alpha_g(f_\beta) = f_\beta$. То есть все коэфиициенты $f_\beta$
    сохраняются под действием любого элемента $H$.

    Поскольку $K^H = F$, все
    коэффициенты $f_\beta$ лежат в $F$, то есть $f_\beta \in F[x]$.

    Поскольку  $id \in H \Rightarrow f_\beta(\beta) = 0 \Rightarrow
    m_{\beta, F} | f_\beta$. То есть, во первых, $\deg m_{\beta, f} \leqslant
    \deg f_\beta = |H|$ (последнее равенство -- из определения $f_\beta$).

    Во-вторых, все корни $m_{\beta, f}$ являются корнями $f_\beta$, то есть
    образами $\beta$ при каком-то автоморфизме $K$, то есть лежат в $K$.

    Значит, все сопряженные к $\beta$ лежат в $K$.

    По условию $K^{Aut_FK} = F$, значит, по утверждению, любой сопряженный к
    любому элементу $K$ лежит в $K$. Что и требовалось доказать.



\hypertarget{9.2}{\subsection{Теорема Гильберта о базисе}}

    Нужно доказать, что если $K$ --- нетерово, то и $K[x]$ тоже нетерово
    (это и есть теорема Гильберта о базисе).

    Пусть есть цепочка строго вложеных в $K[x]$ идеалов
    $I_1 \subsetneq I_2 \subsetneq \ldots  \subsetneq I_n \subsetneq \ldots$

    Положим $I = \cup I_i$. Как неоднократно обсуждалось (\hyperlink{5.6}{5.6},
    \hyperlink{8.2}{8.2}) $I$ --- идеал.

    Будем итеративно строить последовательность
    $f_1, \ldots, f_n, \ldots \in K[x]$

    На $i$-м шаге будем выбирать
    $f_i \in I \backslash (f_1, f_2, \ldots, f_{i - 1}): \deg f_i \to \min$.

    (На первом шаге просто выберем $f_i \in I: \deg f_1 \to \min$.
    Под $(f_1, \ldots, f_{i - 1})$ подразумевается идеал, порожденный
    соответствующими многочленами).

    Корректность выбора (т.е что такое $f_i$ существует) следует из того, что
    $f_1, \ldots f_{i - 1} \in
    I_{i - 1} \Rightarrow (f_1, \ldots, f_{i - 1}) \subset I_{i - 1}
    \subsetneq I_i \subset I$.

    Рассмотрим теперь старшие коэффициенты этих многочленов
    $a_1, a_2, \ldots, a_n, \ldots$. Сразу заметим, что при $i < j:
    I \backslash (f_1, \ldots, f_i) \supset I \backslash (f_1, \ldots, f_j)
    \Rightarrow \deg f_i \leqslant \deg f_j$.

    Рассмотрим цепочку идеалов $(a_1) \subset (a_1, a_2) \subset \ldots
    \subset (a_1, \ldots, a_n) \subset \ldots $. Это последовательность
    вложенных идеалов из $K$. Поскольку $K$ --- нетерово, она стабилизируется,
    то есть существует такое $N$, что $a_{N + 1} \in (a_1, \ldots, a_N)
    \Rightarrow \; \exists \; b_1, b_2, \ldots b_N: a_{N + 1}
    = \sum\limits_{i = 1}^N b_i a_i$.

    Рассмотрим $f = f_{N + 1} - \sum\limits_{i = 1}^N b_i \cdot f_i \cdot
    x^{\deg f_{N + 1} - \deg f_i}$. (Все степени $x$-ов неотрицательны
    по замечанию выше). Степень $f$ строго меньше степени $f_{N + 1}$. С другой
    стороны, если $f \in (f_1, \ldots, f_N) \Rightarrow f_{N + 1} \in
    (f_1, \ldots, f_N)$, что не так. Получили противоречие с минимальностью
    степени $f_{N + 1}$.

    То есть в $K[x]$ не существует последовательности строго вложенных идеалов.

    Пусть в $K[x]$ есть последовательность вложенных идеалов, которая
    не стабилизируется. Тогда из нее можно выделить подпоследовательность
    строго вложенных идеалов. (Не стабилизируется равносильно тому, что
    $\forall N \exists n > N: I_N \subsetneq I_n$).

    Получили, что $K[x]$ нетерово, что и требовалось.


\hypertarget{9.3}{\subsection{Если кольцо $K$ факториально, то $K[x]$
    тоже факториально}}

    Известное всем утверждение: если $K$ --- область целостности, то и
    $K[x]$ --- область целостности, причем $\deg ab \geqslant \deg a, \deg b$.
    ({\HugeСсылка!!!})

    Для начала покажем, что если $p$ неразложим в $K$, то $p$ неразложим
    в $K[x]$. Действительно, пусть $\deg p \leqslant 0, p = ab$. Тогда
    $\deg a, \deg b \leqslant 0$, то есть $a \in K, b \in K$. Но поскольку
    $p$ неразложим в $K$, то $a \in K^* \lor b \in K^*$. А поскольку
    обратимые элементы $K$ --- это в точности обратимые элементы $K[x]$
    (в силу того, что единица одна и та же и соображений степеней), получаем
    требуемое утверждение.

    Теперь покажем, что если $p$ неразложим в $K$, то $p$ прост в $K[x]$.

    Пусть $p | gh$. Посмотрим на $g$ и $h$ как на элементы $(K / (p))[x]$.
    (то есть рассмотрим коэффициенты по модулю $p$).
    (Обозначим их как $\overline{g}$ и $\overline{h}$ соответственно).

    Поскольку $p$ неприводим в $K$ и $K$ факториально, то $p$ прост в $K$
    (\hyperlink{7.3}{7.3}), а значит, $(K / (p))$ --- область целостности
    (\hyperlink{6.9}{6.9}), а значит $(K / (p))[x]$ --- область целостности.

    $p | gh \Rightarrow \overline{g}\overline{h} = 0 \Rightarrow
    \overline{g} = 0 \lor \overline{h} = 0 \Rightarrow p | g \lor p | h$.

    (Тут неявно используется простое утверждение, что $K \ni p | g
    \Leftrightarrow $ все коэффициенты $g$ делятся на $p$: просто вынести $p$
    за скобку или наоборот, внести).

    Теперь пусть $f$ примитивный элемент $K[x]$ (то есть НОД всех его
    коэффициентов равен единице). Пусть $f = g \cdot h$ в
    $\operatorname{Quot}(K)[x]$, причем $\deg g, \deg h \geqslant 1$. Тогда
    существуют такие $\hat{g}, \hat{h} \in K[x]: f = \hat{g} \cdot \hat{h},
    \deg \hat{g}, \deg \hat{h} \geqslant 1$.

    В дальнейших рассуждениях, когда я буду говорить ``числитель'' и
    ``знаменатель'', я буду иметь в виду, что все дроби записаны в несократимом
    виде (то есть что числитель и знаменатель взаимно просты)

    Действительно, пусть $c_g = \frac{\textup{НОД всех числителей g}}
    {\textup{НОК всех знаменателей g}}$.

    Обозначим $\hat{g} = \frac{1}{c_g}g, \hat{h} = \frac{1}{c_h}h$.

    Утверждение: $\hat{g}$ --- примитивный многочлен из $K[x]$.

    Доказательство утверждения: Пусть $a_n, \ldots, a_0$ --- числители $g$,
    $b_n, \ldots, b_0$ --- знаменатели. Обозначим за $(a, b)$ НОД двух
    (или более) чисел, за $[a, b]$ --- НОК.

    Пусть $a_i = (a_0, \ldots, a_n) \cdot a_i',
    b_i' = [b_n, \ldots, b_0] / b_i$.

    $a_0, \ldots, a_n$ делятся на
    $(a_0, \ldots, a_n) \cdot (a_0', \ldots, a_n') \Rightarrow
    (a_0, \ldots, a_n) \cdot (a_0', \ldots, a_n') | (a_0, \ldots, a_n)$
    (поскольку $(a_0, \ldots, a_n)$ --- НОД). Но это значит, что
    $(a_0', \ldots, a_n') = 1$ и значит $a_i'$ взаимно просты.

    $b_i'$ тоже взаимно просты: $b_i | [b_0, \ldots, b_n] / b_i' \Rightarrow
    b_i | [b_0, \ldots, b_n] / (b_0', \ldots, b_n') \forall i \Rightarrow
    [b_0, \ldots, b_n] | [b_0, \ldots, b_n] / (b_0', \ldots, b_n')$
    (в силу определения НОК).

    Теперь покажем, что $a_i' \cdot b_i'$ взаимно просты.
    (Эти числа и будут коэффициентами $\hat{g}$). Пусть они все делятся
    на какое-то необратимое число $p$. В силу факториальности $K$ $p$ можно
    считать простым. Каждое число $a_i' \cdot [b_0, \ldots, b_n] / b_i$
    делится на $p$, значит, в силу определения простоты, для любого $i$
    либо $a_i$ делится на $p$, либо $[b_0, \ldots, b_n] / b_i$ делится на $p$.

    Все $a_i$ одновременно делится на $p$ не могут. Пусть $k$ такое, что
    $b_k$ делится на максимальную степень $p$ (среди $b_i$).
    Пусть $p^l | b_k; p^{l + 1} \nmid b_k$ . Заметим, что
    именно на такую степень делится $[b_0, \ldots, b_n]$
    (меньше не может быть, ведь $b_k | [b_0, \ldots, b_n]$,
    Пусть $p^{l + 1} | [b_0, \ldots, b_n].
    \forall i [b_0, \ldots b_n] = b_i' \cdot b_i$. Поскольку в разложении
    на неразложимые в левой части $p$ входит в хотя бы $p^{l + 1}$ степени, а
    $p^{l + 1} \nmid b_i$, то все $b_i'$ делятся на $p$, что невозможно в силу
    их взаимной простоты).

    Рассмотрим $a_k' \cdot [b_0, \ldots, b_n] / b_k$. С одной стороны,
    $a_k'$ взаимно просто с $b_k$ (так как $a_k$ взаимно просто с $b_k$, а
    $a_k' | a_k$, то есть $p \nmid a_k'$. С другой стороны, в разложение на
    неразложимые $[b_0, \ldots, b_n]$ и $b_k$ $p$ входит в одной и той же
    степени). Значит, $[b_0, \ldots, b_n] / b_k$ не делится на $p$.
    Противоречие.

    Продолжим.

    Тогда $g = c_g \hat{g}$,
    $h = c_h \hat{h}$, где $\hat{g}, \hat{h} \in K[x]$ ---
    примитивные многочлены.

    Пусть $c_g \cdot c_h
    = \frac{u \cdot p_1^{\alpha_1} \cdot \ldots \cdot p_k^{\alpha_k}}
    {v \cdot q_1^{\beta_1} \cdot \ldots \cdot q_l^{\beta_l}}$.
    (Разложили числитель и знаменатель дроби на неразложимые и обратимые.
    Напомню, что мы считаем, что числитель и знаменатель взаимно просты).

    \hypertarget{9.3.no.denominator}{Рассмотрим} $q_1$.
    $q_1 \cdot v \cdot q_1^{\beta_1 - 1} \cdot \ldots \cdot q_l^{\beta_l}
    \cdot f
    = u \cdot p_1^{\alpha_1} \cdot \ldots \cdot p_k^{\alpha_k}
    \cdot \hat{g} \cdot \hat{h}$, то есть
    $u \cdot p_1^{\alpha_1} \cdot \ldots \cdot p_k^{\alpha_k}
    \cdot \hat{g} \cdot \hat{h}$ делится на $q_1$ в $K$. $q_1$ прост в $K$,
    значит он прост в $K[x]$.  Тогда либо
    $u \cdot p_1^{\alpha_1} \cdot \ldots \cdot p_k^{\alpha_k}$ делится на $q_1$
    (что не так в силу взаимной простоты числителя и знаменателя), либо
    $\hat{g}$ делится на $q_1$ либо $\hat{h}$. Но это тоже не так в силу
    примитивности $\hat{g}, \hat{h}$. То есть на самом деле никакого
    знаменателя нет (можно считать, что нет даже ``обратимой'' его части ($v$)
    так как ее всегда можно засунуть в $\hat{g}$, например. Давайте также
    считать, что и $u = 1$).

    Итак, $f = p_1^{\alpha_1} \cdot \ldots \cdot p_k^{\alpha_k} \cdot
    \hat{g} \hat{h}$. В силу примитивности $f$ все $\alpha_i$ равны нулю, так
    что $f = \hat{g} \cdot \hat{h}$. Поскольку $\deg \hat{g} = \deg g,
    \deg \hat{h} = \deg h$ заключаем требуемое.

    То есть мы доказали, что если $f$ --- примитивный элемент $K[x]$, то он
    неразложим тогда и только тогда, когда $f$ неразложим в
    $\operatorname{Quot}(K)[x]$.

    Покажем, что если $f$ неразложим в $K[x]$, то $f$ прост в $K[x]$.

    Если $\deg f \leqslant 0$ то мы это уже показывали. Если $\deg f > 0$ и
    $f$ не примитивный, то он не неразложим ($f$ делится на НОД своих
    коэффициентов). Иначе же $f$ неразложим в $\Quot(K)[x]$, следовательно,
    прост в $\Quot(K)[x]$. (Так как многочлены над полем --- евклидово кольцо).

    Пусть $f | gg_1$ в $K[x]$. Тогда $f | gg_1$ в $\Quot(K)[x]$, и значит
    $f | g \lor f | g_1$. Пусть, без потери общности, $f | g \Rightarrow
    fh = g$ в $\Quot(K)$. Заметим, что $f, g \in K[x]$. Покажем, что
    $h \in K[x]$.

    $h = c_h \cdot \hat{h}$, где $\hat{h}$ --- примитивный. $f$ тоже примитивный
    и, значит, у $c_h$ нет знаменателя
    (\hyperlink{9.3.no.denominator}{рассуждение} недавно проводилось выше ---
    пусть есть, возьмем простой делитель $\ldots$), то есть $h \in K[x]$.

    Таким образом мы показали, что любой неразложимый элемент $K[x]$ прост.

    Покажем существование разложения индукцией по степени.

    Если $\deg f \leqslant 0$ то разложение совпадает с разложением в $K$.

    Пусть $\deg f \ne 0$. Тогда $f = c_f \cdot \hat{f}$,
    где $\hat{f}$ --- примитивный. У $c_f$ есть разложение. Пусть
    $\hat{f}$ разложим, тогда $\hat{f} = gh$, где $\deg g, \deg h
    < \deg \hat{f} = \deg f$ (в силу примитивности $\hat{f}$), и значит,
    у $g, h$ существуют разложения по предположению индукции.
    Перемножая разложения $c_f, g, h$ получим разложение $f$ в неразложимые.

    Остается воспользоваться утверждением \hyperlink{7.2}{7.2} и
    требуемое доказано.

\hypertarget{9.4}{\subsection{Основная теорема теории Галуа}}
    \textbf{Теорема.} (Основная теорема теории Галуа)
    Пусть $K \supset F$ --- расширение Галуа. Тогда:

    1) Существует биективное соответствие между подполями
    $K \supset L \supset F$ и подгруппами $Aut_FK$, задаваемое отображениями:

    $\varphi: L \mapsto Aut_LK$ и $\psi: H \mapsto K^H$

    2) $L \supset F$ нормальное тогда и только тогда, когда $Aut_LK$ нормальна
    в $Aut_FK$.

    3) $[L : F] = [Aut_FK : Aut_LK]$

    Докажем ее:

    1) Заметим, что если $K \supset L \supset F$, то
    $K \supset L$ --- расширение Галуа, поскольку $K$ является
    полем разложения некоторого многочлена $f$ над $F$ (в силу нормальности
    $K \supset F$), а значит, $K$ является полем разложения $f$ над $L$ ($f$
    раскладывается к $K$ на линейные множители,
    а если есть какое-то промежуточное поле $K \supset K' \supset L$, что
    в $K'$ $f$ раскладывается на линейные множители, то существует
    $K \supset K' \supset F$ с нужными свойствами, что противоречит тому, что
    $K$ --- поле разложения $f$ над $F$)

    Итак, $\psi(\varphi(L)) = \psi(Aut_LK) = K^{Aut_LK} = L$ в силу
    \hyperlink{9.1}{3-го определения расширения Галуа}.

    Пусть $\psi(H) = K^H =: L$, тогда пусть
    $\varphi(\psi(H)) = Aut_LK =: H'$. Заметим, что из определения $L$
    $H \subset H'$. Поскольку $K \supset L$ --- расширение Галуа, а значит
    конечное и сепарабельное, можно применить теорему о примитивном элементе
    (или ее конечный аналог --- теорему о цикличности мультипликативной
    группы поля), то есть $K = L(\gamma)$.

    С другой стороны, по
    \hyperlink{9.1.useful.statement}{утверждению из доказательства 9.1},
    примененного к расширению $K \supset L$ и $H: |H| \geqslant
    \deg m_{\gamma, L} = [K : L] = |H'|$. То есть на самом деле $H' = H$.

    Таким образом, $\varphi$ и $\psi$ --- взаимно обратные преобразования,
    а значит биекции.

    2) Возьмем $g \in Aut_FK$. $K^{gHg^{-1}} = \{x \in K \; |
    \; \forall h \in H ghg^{-1}(x) = x \} = \{x \in K \; |
    \; \forall h \in H hg^{-1}(x) = g^{-1}(x) = \{x \in gK \; |
    \forall h \in H h(x) = x\} = g K^H$

    То есть $H \triangleleft Aut_FK \Leftrightarrow
    \forall g \in Aut_FK \; gHg^{-1} = H
    \stackrel{\varphi \textup{ --- биекция}}
    {\Leftrightarrow} \forall g \in Aut_FK K^H = K^{gHg^{-1}} = g K^H$

    Покажем, что $\forall g \in Aut_FK K^H = g K^H \Leftrightarrow K
    \supset K^H$ --- нормальное.

    $\Rightarrow$

    Пусть $\forall g \in Aut_FK \; K^H = gK^H$. Пусть $\alpha \in K^H$.
    Поскольку
    группа Галуа $Aut_FK$ действует транзитивно на корнях $m_{\alpha, F}$,
    для любого $\beta$ сопряженного с $\alpha$ над $F$ существует
    $g \in Aut_FK: g(\alpha) = \beta$. Но $\alpha \in K^H \Rightarrow
    \beta \in g(K^H) = K^H$, то есть все сопряженные к любому элементу $K^H$
    лежат в $K^H$, то есть $K^H \supset F$ --- нормальное.

    $\Leftarrow$

    Поскольку $\forall \alpha \in K^H, \forall g \in Aut_FK: g(\alpha)$ сопряжен
    к $\alpha$, а все сопряженные к $\alpha$ элементы лежат в $K^H$ поскольку
    $K^H \supset F$ --- нормальное, то $g(K^H) \subset K^H \forall g
    \in Aut_FK$.

    Но тогда $g^{-1}(K^H) \subset K^H \Rightarrow g(K^H) = K^H$.

    Что и требовалось доказать.

    3) $[L : F] = [K : F] / [K : L] = |Aut_FK| / |Aut_LK| =  [Aut_FK : Aut_LK]$.
    Второе равенство выполнено, так как $K \supset L$ и $K \supset F$
    --- расширения Галуа.

    4) \hypertarget{9.4.bonus}{(Бонус)}
    Если $K \supset L \supset F$, и $K \supset F$, $L \supset F$
    нормальные, то $Aut_FL \cong \bigslant{Aut_FK}{Aut_LK}$.

    Доказательство: Построим гомоморфизм $\varphi: Aut_FK \to Aut_FL$ следующим
    образом: $\varphi(g) = g|_L$. Это определение корректно, так как
    $g|_L$ --- гомоморфизм из $L$ в $\overline{F}$, а значит,
    \hyperlink{9.1.statement.2}{по утверждению из доказательства 9.1},
    $g|_L$ --- автоморфизм $L$. Ядро же этого гомоморфизма, очевидно $Aut_LK$.

    Применим основную теорему о гомоморфизмах:
    $Aut_FL \gtrsim \bigslant{Aut_FK}{Aut_LK}$. Но по пункту 3 порядки этих групп
    равны, то есть $Aut_FL \cong \bigslant{Aut_FK}{Aut_LK}$, что и требовалось.

\hypertarget{9.5}{\subsection{Основная теорема алгебры}}

    \textbf{Теорема.} $\mathbb{C}$ --- алгебраически замкнутое поле.

    Нам понадобятся два следующих утверждения:

    1) Над $\mathbb{R}$ не бывает нетривиальных конечных расширений нечетной
    степени.

    Доказательство: Пусть $K \supset \mathbb{R}$ --- конечное расширение.

    По теореме о примитивном элементе $K = \mathbb{R}(\gamma)$.
    $\deg m_{\gamma, \mathbb{R}} = [K : \mathbb{R}]$. Если $[K : \mathbb{R}]$
    нечетно, то по известному факту из анализа, $m_{\gamma, \mathbb{R}}$ имеет
    корень. Но тогда, в силу неприводимости, его степень равна единице, то есть
    расширение --- тривиально.

    2) Над $\mathbb{C}$ не существует расширений второй степени.

    Пусть $K \supset \mathbb{C}$ --- расширение второй степени, то есть

    $K = \mathbb{C}(\gamma)$, где $\deg m_{\gamma, \mathbb{C}} = 2$. Но над
    $\mathbb{C}$ не бывает неприводимых многочленов второй степени
    (поскольку можно найти корни через формулу с дискриминантом и разложить
    по теореме Виетта на два линейных сомножителя).

    Теперь, пусть над $\mathbb{C}$ есть нетривиальное алгебраическое расширение
    $K_1$. Выберем $\gamma \in K_1 \backslash \mathbb{C}$.
    $\mathbb{C}(\gamma) \supset \mathbb{C} \supset \mathbb{R}$ ---
    башня конечных алгебраических расширеший. Рассмотрим поле разложения
    $m_{\gamma, \mathbb{R}}$ над $\mathbb{C}$.

    \textbf{Лемма.} Пусть $K \supset L \supset F$ и $L \supset F$ ---
    нормальное, а $K$ является полем разложения $f \in F[x]$ над $L$.
    (Соответственно, $K \supset L$ нормально). Тогда $K \supset F$
    --- нормально.

    Пусть $\alpha_1, \ldots, \alpha_n$ --- корни $f$, тогда
    $K = L(\alpha_1, \ldots, \alpha_n)$. Пусть $L$ является полем разложения
    $g$ над $F$, и корнями $g$ являются $\beta_1, \ldots, \beta_m$. Тогда
    $L = F(\beta_1, \ldots, \beta_m)$. Тогда $K = F(\alpha_1, \ldots, \alpha_n,
    \beta_1, \ldots, \beta_m)$, и, значит, $K$ является полем разложения $fg$
    над $F$. То есть $K \supset F$ --- нормальное.

    Продолжим.

    Итак, $\mathbb{C} \supset \mathbb{R}$ нормально (как поле разложения
    $x^2 + 1$), и $K$ является полем разложения многочлена с коэффициентами
    из $\mathbb{R}$ над $\mathbb{C}$. (Многочлен --- $m_{\gamma, \mathbb{R}}$).

    То есть $K$ нормально над $\mathbb{R}$. Пусть $[K : \mathbb{C}] = t$, и
    $t = 2^{n - 1} \cdot m$, где $(m, 2) = 1$.

    Пусть также $G = Aut_{\mathbb{R}}K$. Так как расширение нормально,
    $|G| = [K : \mathbb{R}] = 2^n \cdot m$. По теореме Силова в $|G|$ есть
    подгруппа порядка $2^n$. Ей соответствует некоторое подполе
    $L: K \supset L \supset \mathbb{R}$, причем $[L : \mathbb{R}] = m$
    (по основной теореме теории Галуа). Но так как $m$ нечетно, то по
    первому утверждению $m = 1$.

    То есть $[K : C] = 2^{n - 1}$. Это расширение также нормально, пусть
    $H$ --- его группа Галуа. Тогда в ней есть подгруппа порядка $2^{n - 2}$
    (если $n \geqslant 2$) [Это факт из ТГ, например, следует из доказательства
    теоремы Силова, приведенного в Кострикине]. Тогда есть
    соответствующее ей подполе $L: K \supset L \supset \mathbb{C}$, причем
    $[L : \mathbb{C}] = 2$, чего не бывает. То есть $n = 1$ и $K = \mathbb{C}$,
    то есть над $\mathbb{C}$ нет нетривиальных алгебраических расширений.
    Что и требовалось доказать.

\hypertarget{9.6}{\subsection{Теорема Ферма при $n = 3$ с использованием чисел Эйзенштейна}}

\hypertarget{9.7}{\subsection{Сведение разрешимости уравнения в радикалах к разрешимости соответствующей группы Галуа}}
(Теоремой Куммера можно пользоваться без доказательства)
\hypertarget{9.8}{\subsection{Пример уравнения, неразрешимого в радикалах}}
(Теоремой о разрешимости группы Галуа можно пользоваться без доказательства).

\hypertarget{9.9}{\subsection{Неприводимость многочлена деления круга $\Psi(x)$ над $\mathbb{Q}$}}

\hypertarget{9.10}{\subsection{Теорема Островского}}


\end{document}
