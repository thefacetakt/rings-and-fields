\documentclass[../main.tex]{subfiles}

\begin{document}

\section{Вопросы на 9}

    \hypertarget{9.1}{\subsection{Эквивалентность определений нормального сепарабельного расширения
    (расширения Галуа)}}.

    Пусть $K \supset F$ --- конечное сепарабельное расширение.
    Тогда следуещие условия эквивалентны:

    \begin{enumerate}
        \item Для любого элемента $\alpha \in K$, любой сопряженный к $\alpha$
        над $F$ тоже лежит в $K$

        \item $K$ является полем разложения какого-либо многолена над $F$

        \item $|Aut_FK| = [K : F]$

        \item $K^{Aut_FK} = F$
    \end{enumerate}

    Такое расширение называется \textit{нормальным}
    или \textit{расширением Галуа}.

    $1 \Rightarrow 2$

    Так как расширение конечное, $K = F(\alpha_1, \ldots, \alpha_n)$.

    Положим $f := m_{\alpha_1, F} \cdot \ldots \cdot m_{\alpha_n, F}$.
    Тогда, поскольку все сопряженные к $\alpha_1, \ldots, \alpha_n$
    лежат в $K$, $K$ содержит все корни $f$. С другой стороны, если
    $K \supset L$ содержит все корни $F$, то $\alpha_1, \ldots, \alpha_n
    \in L \Rightarrow F(\alpha_1, \ldots, \alpha_n) \subset L \Rightarrow
    K \subset L \subset K \Rightarrow L = K$,
    то есть $K$ --- поле разложения $f$ над $F$.

    $2 \Rightarrow 3$

    \textbf{Утверждение 1.} Любой гомоморфизм $\varphi K \to \overline{F}$,
    сохраняющий $F$ переводит элементы $K$ в сопряженные к ним над $F$.

    Доказательство утверждения 1:

    Пусть $\alpha \in K, m_{\alpha, F} = \sum\limits_{k=0}^n a_kx^k.
    m_{\alpha, F}(\alpha) = 0 \Rightarrow \varphi(m_{\alpha, F}(\alpha))
    = \varphi(0) = 0$.

    С другой стороны $0 = \varphi(m_{\alpha, F}(\alpha))
    = \varphi(\sum\limits_{k=0}^n a_k \alpha^k)
    = \sum\limits_{k=0}^n a_k \varphi(\alpha)^k
    = m_{\alpha, F}(\varphi(\alpha))$,
    что и означает, что $\varphi(\alpha)$ сопряжен к $\alpha$ над $F$.

    \textbf{Утверждение 2.} Пусть $\varphi: K \to \overline{F}$
    --- гомоморфизм, сохраняющий $F$.
    Тогда $\varphi$ является автоморфизмом $K$.

    Действительно, пусть $K$ --- поле разложения $f$ над $F$, и
    $\alpha_1, \ldots, \alpha_n$ --- корни $f$.

    Тогда $K = F(\alpha_1, \ldots, \alpha_n)$.

    Поскольку для любого $i: m_{\alpha_i, F} | f \Rightarrow$
    то все сопряженные к $\alpha_i$ над $F$ находятся среди корней $f$.

    По утверждению 1 множество $\{\alpha_1, \ldots, \alpha_n\}$ переходит в свое
    подмножество, а учитывая, что любой нетривиальный гомоморфизм полей
    инъективен, то на самом деле оно переходит само в себя (в силу конечности).
    Тогда $\varphi$ задает на множестве индексов корней $f$ некую перестановку
    $\sigma$.

    Пусть $\beta \in K, \beta = \sum\limits_{k=0}^n a_k \alpha_k, a_k \in F$.
    Тогда $\varphi(\beta) = \varphi(\sum\limits_{k=0}^n a_k \alpha_k) =
    = \sum\limits_{k=0}^n a_k \alpha_{\sigma(k)} \in K$.

    То есть $\varphi(K) \subset K$.

    С другой стороны $\varphi(\sum\limits_{k = 0}^n a_k \alpha_{\sigma^{-1}(k)})
    = \sum\limits_{k = 0}^n a_k \alpha_k = \beta$. То есть $\varphi(K) = K$.

    Итого, $\varphi:K \to K$ --- сюръективный гомоморфизм полей, а значит
    --- автоморфизм $K$.

    Поскольку $K \supset F$ --- конечное сепарабельное, то по теореме о
    примитивном элементе найдется такое $\gamma$, что $K = F(\gamma)$.

    Пусть $\gamma = \gamma_1, \gamma_2, \ldots, \gamma_m$ --- корни
    $m_{\gamma, F}$.

    Вспомним утверждение \hyperlink{6.13}{6.13} (точнее, его доказательство).

    $K = F(\gamma_1) \stackrel{\varphi}{\cong} F(\gamma_i)$, причем $\varphi$
    сохраняет $F$ и $\varphi(\gamma_1) = \gamma_2$.

    $\varphi: K \to \overline{F}$ --- гомоморфизм, сохраняющий $F$,
    следовательно, по утверждению 2 он является автоморфизмом $K$, сохраняющем
    $F$. То есть для любого $i$ существует $\varphi \in Aut_FK:
    \varphi(\gamma_1) = \gamma_i \Rightarrow |Aut_FK| \geqslant m$.

    С другой стороны, тем, куда переходит $\gamma = \gamma_1$ автоморфизм,
    сохраняющий $F$ полностью определяется (поскольку любой элемент $K$
    разлагается по степеням $\gamma$ с коэффициентами из $F$). Значит,
    $Aut_FK \leqslant m$. Значит, $Aut_FK = m = \deg m_{\gamma, F} = [K : F]$,
    что и требовалось доказать.

    $3 \Rightarrow 4$

    Пусть $K^{Aut_FK} = L. K \supset L \supset F$ (\hyperlink{5.31}{5.31})

    Пусть, по-прежнему, $K = F(\gamma)$. Мы уже выяснили, что при автоморфизме
    $K$, сохраняющем $F$ $\gamma$ переходит в корень $m_{\gamma, F}$, причем
    тем, куда переходит $\gamma$ полностью определяется автоморфизм.

    Заметим, что $K = L(\gamma)$, и что все вышесказанное справедливо и для
    расширения $K \supset L$, то есть $|Aut_LK| \leqslant \deg m_{\gamma, L}
    = [K : L]$

    Все автоморфизмы, сохраняющие $F$ сохраняют и $L$ (по определению $L$),
    значит, $|Aut_FK| \leqslant |Aut_LK| \leqslant [K : L] \leqslant [K : F]$.

    Но $|Aut_FK| = [K : F] \Rightarrow [K : L] = [K : F] \Rightarrow L = F$.

    $4 \Rightarrow 1$

    Рассмотрим вспомогательное утверждение:

    Пусть $K^H = F$. Тогда для любого $\beta \in K: |H| \geqslant
    m_{\alpha, F}$ (это нам конкретно сейчас не понадобится)
    и любой сопряженный к $\beta$ над $F$ лежит в $K$
    (а вот это будем использовать).

    Докажем его. Рассмотрим $f_\beta
    = {\displaystyle \prod_{h \in H}(x - h(\beta))}$.

    Рассмотрим действие элементами $H$ на элементах
    $K[x]:$

    $H \ni h \mapsto \alpha_h(\sum a_k x^k) = \sum h(a_k) x^k$.

    Проверим, что это действие
    (напомню: действие, это гомомофизм из $H$ в группу биекций $K[x]$).

    1) Инъективность:

    Пусть $\alpha_h(g_1) = \alpha_h(g_2)$, тогда образы всех
    коэффициенто $g_1$ совпадают с образами всех коэффициентов $g_2$.
    Но $h$ --- автоморфизм, так что все \textit{коэффициенты} $g_1$ совпадают
    с коэффициентами $g_2$

    2) Сюръективность:

    $\alpha_h(\sum h^{-1}(\alpha_k) x^k) = \sum \alpha_kx^k$

    3) Гомоморфность:

    $\alpha_{h_1} \circ \alpha_{h_2}(\sum a_k x^k)
    = \alpha_{h_1}(\sum h_2(a_k) x^k) = \sum h_1 h_2 (a_k) x^k
    = \alpha_{h_1 h_2}$

    Заметим также, что $\alpha_h((\sum a_k x^k)(\sum b_kx^k))
    = \alpha_h(\sum (\sum\limits_{i + j = k} a_ib_j)x^k)
    = \sum h(\sum\limits_{i + j = k} a_ib_j)x^k
    = \sum (\sum\limits_{i + j = k} h(a_i)h(b_j)) x^k
    = (\sum h(a_k)x^k)(\sum h(b_k)x^k)
    = \alpha_h(\sum a_kx^k)\alpha_h(\sum b_kx^k)$.

    Иными словами, $\alpha_h(fg) = \alpha_h(f)\alpha_h(g)$.

    Возьмем произвольный $g \in H$. Учитывая вышесказанное

    $\alpha_g(f_\beta) = \prod\limits_{h \in H} \alpha_g(x - h(\beta))
    =  \prod\limits_{h \in H} (x - gh(\beta))$. Но умножение на элемент
    группы есть автоморфизм группы, то есть $gH = H$, то есть
    $\alpha_g(f_\beta) = f_\beta$. То есть все коэфиициенты $f_\beta$
    сохраняются под действием любого элемента $H$.

    Поскольку $K^H = F$, все
    коэффициенты $f_\beta$ лежат в $F$, то есть $f_\beta \in F[x]$.

    Поскольку  $id \in H \Rightarrow f_\beta(\beta) = 0 \Rightarrow
    m_{\beta, F} | f_\beta$. То есть, во первых, $\deg m_{\beta, f} \leqslant
    \deg f_\beta = |H|$ (последнее равенство -- из определения $f_\beta$).

    Во-вторых, все корни $m_{\beta, f}$ являются корнями $f_\beta$, то есть
    образами $\beta$ при каком-то автоморфизме $K$, то есть лежат в $K$.

    Значит, все сопряженные к $\beta$ лежат в $K$.

    По условию $K^{Aut_FK} = F$, значит, по утверждению, любой сопряженный к
    любому элементу $K$ лежит в $K$. Что и требовалось доказать.



\hypertarget{9.2}{\subsection{Теорема Гильберта о базисе}}

    Нужно доказать, что если $K$ --- нетерово, то и $K[x]$ тоже нетерово
    (это и есть теорема Гильберта о базисе).

    Пусть есть цепочка строго вложеных в $K[x]$ идеалов
    $I_1 \subsetneq I_2 \subsetneq \ldots  \subsetneq I_n \subsetneq \ldots$

    Положим $I = \cup I_i$. Как неоднократно обсуждалось (\hyperlink{5.6}{5.6},
    \hyperlink{8.2}{8.2}) $I$ --- идеал.

    Будем итеративно строить последовательность
    $f_1, \ldots, f_n, \ldots \in K[x]$

    На $i$-м шаге будем выбирать
    $f_i \in I \backslash (f_1, f_2, \ldots, f_{i - 1}): \deg f_i \to \min$.

    (На первом шаге просто выберем $f_i \in I: \deg f_1 \to \min$.
    Под $(f_1, \ldots, f_{i - 1})$ подразумевается идеал, порожденный
    соответствующими многочленами).

    Корректность выбора (т.е что такое $f_i$ существует) следует из того, что
    $f_1, \ldots f_{i - 1} \in
    I_{i - 1} \Rightarrow (f_1, \ldots, f_{i - 1}) \subset I_{i - 1}
    \subsetneq I_i \subset I$.

    Рассмотрим теперь старшие коэффициенты этих многочленов
    $a_1, a_2, \ldots, a_n, \ldots$. Сразу заметим, что при $i < j:
    I \backslash (f_1, \ldots, f_i) \supset I \backslash (f_1, \ldots, f_j)
    \Rightarrow \deg f_i \leqslant \deg f_j$.

    Рассмотрим цепочку идеалов $(a_1) \subset (a_1, a_2) \subset \ldots
    \subset (a_1, \ldots, a_n) \subset \ldots $. Это последовательность
    вложенных идеалов из $K$. Поскольку $K$ --- нетерово, она стабилизируется,
    то есть существует такое $N$, что $a_{N + 1} \in (a_1, \ldots, a_N)
    \Rightarrow \; \exists \; b_1, b_2, \ldots b_N: a_{N + 1}
    = \sum\limits_{i = 1}^N b_i a_i$.

    Рассмотрим $f = f_{N + 1} - \sum\limits_{i = 1}^N b_i \cdot f_i \cdot
    x^{\deg f_{N + 1} - \deg f_i}$. (Все степени $x$-ов неотрицательны
    по замечанию выше). Степень $f$ строго меньше степени $f_{N + 1}$. С другой
    стороны, если $f \in (f_1, \ldots, f_N) \Rightarrow f_{N + 1} \in
    (f_1, \ldots, f_N)$, что не так. Получили противоречие с минимальностью
    степени $f_{N + 1}$.

    То есть в $K[x]$ не существует последовательности строго вложенных идеалов.

    Пусть в $K[x]$ есть последовательность вложенных идеалов, которая
    не стабилизируется. Тогда из нее можно выделить подпоследовательность
    строго вложенных идеалов. (Не стабилизируется равносильно тому, что
    $\forall N \exists n > N: I_N \subsetneq I_n$).

    Получили, что $K[x]$ нетерово, что и требовалось.


\hypertarget{9.3}{\subsection{Если кольцо $K$ факториально, то $K[x]$ тоже факториально}}

\hypertarget{9.4}{\subsection{Основная теорема теории Галуа}}

\hypertarget{9.5}{\subsection{Основная теорема алгебры}}

\hypertarget{9.6}{\subsection{Теорема Ферма при $n = 3$ с использованием чисел Эйзенштейна}}

\hypertarget{9.7}{\subsection{Сведение разрешимости уравнения в радикалах к разрешимости соответствующей группы Галуа}}
(Теоремой Куммера можно пользоваться без доказательства)
\hypertarget{9.8}{\subsection{Пример уравнения, неразрешимого в радикалах}}
(Теоремой о разрешимости группы Галуа можно пользоваться без доказательства).

\hypertarget{9.9}{\subsection{Неприводимость многочлена деления круга $\Psi(x)$ над $\mathbb{Q}$}}

\hypertarget{9.10}{\subsection{Теорема Островского}}


\end{document}
